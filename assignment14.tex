\documentclass[12pt,a4j]{jarticle}
\usepackage[dvipdfmx]{graphicx}
\begin{document}
\title{基礎プログラミングおよび演習 レポート #14}
\author{学籍番号, 氏名 (ペア: 氏名・学籍番号)}
\date{提出日付}
\maketitle

\section{構想・計画・設計}

% (どのような構想で絵を生成したか、プログラムはどう設計したか)

\section{プログラムコード}

% (vervatim環境の中にプログラムのソースコードを入れる)

\subsection{img.h}
\begin{verbatim}
#define WIDTH 300
#define HEIGHT 200
struct color { unsigned char r, g, b; };
...
\end{verbatim}

\subsection{xxx.c}
\begin{verbatim}
#include <stdio.h>
#include <stdlib.h>
#include "img.h"
...

\end{verbatim}

\section{プログラムの説明}

% (プログラムのどの部分が何をしているかを説明する)

\section{生成された動画の説明}

% (画像を入れたい場合は下記で。mypicture.epsというファイル名は適宜変更)
\begin{center}
\includegraphics[width=12cm]{mypicture.eps}
\end{center}

% (どのような動画という説明を書く。)
% (動画ファイルはアップロードで提出。プログラムコードと動画が一致していること。)

\section{開発過程の説明}

% (誰が何を分担し、どのような過程を経てプログラムが完成したか。各作業の日時と担当者の記録があるとよい。)

\section{考察}

% (考察は必須かつ重要。課題をやって分かったこと、理解したことを整理して書く。)

\section{アンケート}

\subsection{Q1:うまく分担して課題プログラムを開発できましたか。}

% (ここにQ1の回答を記入)

\subsection{Q2:複数で分担する際に注意すべきことは何だと思いましたか。}

% (ここにQ2の回答を記入)

\subsection{Q3:ここまでの科目全体を通して、学べたこと、学びたかったけど学べなかったことは何ですか。その他感想や、この科目の今後改善した方がよいこと、今後も維持したことがよいこ との指摘もどうぞ。}

% (ここにQ3の回答を記入)

\end{document}
