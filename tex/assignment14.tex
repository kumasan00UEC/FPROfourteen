\documentclass[12pt,a4j]{jarticle}
\usepackage[dvipdfmx]{graphicx}
\usepackage{listings}
\usepackage{jlisting}


\lstset{
  basicstyle={\ttfamily},
  identifierstyle={\small},
  commentstyle={\smallitshape},
  keywordstyle={\small\bfseries},
  ndkeywordstyle={\small},
  stringstyle={\small\ttfamily},
  frame={tb},
  breaklines=true,
  columns=[l]{fullflexible},
  numbers=left,
  xrightmargin=0zw,
  xleftmargin=3zw,
  numberstyle={\scriptsize},
  stepnumber=1,
  numbersep=1zw,
  lineskip=-0.5ex
}

\begin{document}
\title{基礎プログラミングおよび演習 レポート #14}
% \author{2512125, 角谷拓武 (2512145: 田中琉登, 2512225: 宮崎昭徳)}
% \author{2512145, 田中琉登 (2512125: 角谷拓武, 2512225: 宮崎昭徳)}
% \author{2512225, 宮崎昭徳 (2512125: 角谷拓武, 2512145: 田中琉登)}
\maketitle

\section{構想・計画・設計}

% (どのような構想で絵を生成したか、プログラムはどう設計したか)

\section{プログラムコード}

% (vervatim環境の中にプログラムのソースコードを入れる)

\lstinputlisting[caption=img.c,label=imgc]{../img.c}
\lstinputlisting[caption=img.h,label=imgh]{../img.h}
\lstinputlisting[caption=animate1.c,label=animate1.c]{../animate1.c}


\section{プログラムの説明}

% (プログラムのどの部分が何をしているかを説明する)

\section{生成された動画の説明}

% (画像を入れたい場合は下記で。mypicture.epsというファイル名は適宜変更)
% \begin{center}
% \includegraphics[width=12cm]{mypicture.eps}
% \end{center}

% (どのような動画という説明を書く。)
% (動画ファイルはアップロードで提出。プログラムコードと動画が一致していること。)

\section{開発過程の説明}

% (誰が何を分担し、どのような過程を経てプログラムが完成したか。各作業の日時と担当者の記録があるとよい。)

\section{考察}

% (考察は必須かつ重要。課題をやって分かったこと、理解したことを整理して書く。)

\section{アンケート}

\subsection{Q1:うまく分担して課題プログラムを開発できましたか。}

% (ここにQ1の回答を記入)

\subsection{Q2:複数で分担する際に注意すべきことは何だと思いましたか。}

% (ここにQ2の回答を記入)

\subsection{Q3:ここまでの科目全体を通して、学べたこと、学びたかったけど学べなかったことは何ですか。その他感想や、この科目の今後改善した方がよいこと、今後も維持したことがよいこ との指摘もどうぞ。}

% (ここにQ3の回答を記入)

\end{document}
